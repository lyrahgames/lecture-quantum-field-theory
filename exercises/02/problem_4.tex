\documentclass[crop=false,fleqn]{standalone}
\usepackage{standard}
\usepackage{multicol}

\allowdisplaybreaks

\begin{document}
  \section*{Problem 4}
  % \begin{multicols}{2}
  If not defined otherwise we will use the Latin alphabet $\set{a,b,c,\ldots}{}$ as a set of dummy indices in the set $I_0\define\set{0,1,2,3}{}$ and the Greek alphabet $\set{α,β,γ,\ldots}{}$ as a set of dummy indices in the set $I\define\set{1,2,3}{}$ for the sum convention of Einstein.
  We denote the Minkowski metric as follows.
  \[
    η \define \roundBrackets{η_{mn}} \define \roundBrackets{η^{mn}} \define \diagonalMatrix{-1,1,1,1}
  \]
  Let $\function{A^n}{\setReal^4}{\setReal}$ be a two times continuous differentiable function that denotes one of the electrodynamic potentials and $\function{j^n}{\setReal^4}{\setReal}$ a continuous function which denotes one of the quadruplet currents.
  \[
    A \define \roundBrackets{A^n} \reverseDefine \roundBrackets{φ,\vector{A}}
    \separate
    j \define \roundBrackets{j^n} \reverseDefine \roundBrackets{ρ,\vector{j}}
  \]
  The electric and magnetic field can then be computed as follows.
  \[
    \vector{E} \define -\nabla φ - \partial_t \vector{A}
    \separate
    \vector{B} \define \nabla \times \vector{A}
  \]
  We define the field strength tensor $F$ and its dual $\tilde{F}$ as follows.
  \[
    F_{mn} \define \partial_mA_n - \partial_nA_m
    \separate
    \tilde{F}^{mn} \define \frac{1}{2}ε^{mnpq}F_{pq}
  \]

  \paragraph{(1):}
  For the formulation of the Euler-Lagrange equations it seems to be useful to define the following tensor function.
  \[
    \function{\mathscr{X}_{ij}}{\setReal^{4\times 4}}{\setReal}
    \separate
    \mathscr{X}_{ij}(x) \define x_{ij} - x_{ji}
  \]
  \[
    \mathscr{X}(x) \define \roundBrackets{\mathscr{X}_{ij}(x)}_{i,j\in I_0} = x - \transpose{x}
  \]
  Now we can write the field strength tensor in the following form.
  \[
    F_{mn} = \mathscr{X}_{mn}\circ \transpose{\roundBrackets{\jacobian A}} = \mathscr{X}_{mn}\circ \roundBrackets{\partial_iA_j}_{i,j\in I_0}
  \]
  Then we formulate the Lagrangian density.
  \[
    \function{\mathscr{L}}{\setReal^4\times\setReal^4\times\setReal^{4\times 4}}{\setReal}
  \]
  \[
    \mathscr{L}\roundBrackets{x,a,d} \define -\frac{1}{4}\mathscr{X}_{mn}(d)\mathscr{X}^{mn}(d) + a_mj^m(x)
  \]
  The action $\mathscr{S}(A)$ based on the given Lagrangian can then be computed by the following equation.
  \[
    \mathscr{S}(A) \define \integral{\setReal^4}{}{\mathscr{L}\roundBrackets{x,A(x),\transpose{\jacobian A}(x)}}{λ(x)}
  \]
  Therefore now we can determine the Euler-Lagrange equations.
  \begin{align*}
    0 &= \leibnizPartialDerivativeOperatorValue{x_k}{\mathscr{L}(\cdot,x,\cdot)}{\roundBrackets{\cdot,A,\transpose{\jacobian A}}} -
    \partial_m\boxBrackets{
      \leibnizPartialDerivativeOperatorValue{x_{mk}}{\mathscr{L}(\cdot,\cdot,x)}{\roundBrackets{\cdot,A,\transpose{\jacobian A}}}
    }
  \end{align*}
  The first part can be easily computed because there is only a linear dependence.
  \[
    \leibnizPartialDerivativeOperatorValue{x_k}{\mathscr{L}(\cdot,x,\cdot)}{\roundBrackets{\cdot,A,\transpose{\jacobian A}}} = j^k
  \]
  For the second part we will rely on the sum convention.
  \begin{align*}
    &\leibnizPartialDerivativeOperatorValue{x_{ab}}{\mathscr{L}(\cdot,\cdot,x)}{\roundBrackets{\cdot,A,\transpose{\jacobian A}}} \\
    &= -\frac{1}{4}
    \leibnizPartialDerivativeOperatorValue{x_{ab}}{
      \mathscr{X}_{mn}(x)\mathscr{X}^{mn}(x)
    }{\transpose{\jacobian A}} \\
    &= -\frac{1}{4}\leibnizPartialDerivativeOperatorValue{x_{ab}}{
      (x_{mn}-x_{nm})η^{mp}η^{nq}(x_{pq} - x_{qp})
    }{\transpose{\jacobian A}} \\
    &=
    \appendValue{
      -\frac{1}{4} η^{mp} η^{nq} \boxBrackets{\roundBrackets{δ^a_m δ^b_n - δ^a_n δ^b_m}\roundBrackets{x_{pq} - x_{qp}} + \roundBrackets{x_{mn} - x_{nm}}\roundBrackets{δ^a_p δ^b_q - δ^a_q δ^b_p}}
    }{x = \transpose{\jacobian A}} \\
    &=
    \appendValue{
      -\frac{1}{4}
      \boxBrackets{\roundBrackets{η^{ap}η^{bq} - η^{bp}η^{aq}}\roundBrackets{x_{pq} - x_{qp}} + \roundBrackets{x_{mn} - x_{nm}}\roundBrackets{η^{ma}η^{nb} - η^{mb}η^{na}}}
    }{x = \transpose{\jacobian A}} \\
    &=
    \appendValue{
      -\frac{1}{2}
      \roundBrackets{η^{ap}η^{bq} - η^{bp}η^{aq}}\roundBrackets{x_{pq} - x_{qp}}
    }{x = \transpose{\jacobian A}} \\
    &=
    \appendValue{
      -\roundBrackets{x^{ab} - x^{ba}}
    }{x = \transpose{\jacobian A}} \\
    &= \partial^bA^a - \partial^aA^b = -F^{ab}
  \end{align*}
  Now the Euler-Lagrange equations can be rewritten.
  \[
    \partial_a F^{ka} = j^k
  \]

  \paragraph{(2):}
  We already know that $F$ is given by the following matrix.
  \[
    \roundBrackets{F^{mn}}_{m,n\in I_0} =
    \begin{pmatrix}
      0 & E_x & E_y & E_z \\
      -E_x & 0 & B_z & -B_y \\
      -E_y & -B_z & 0 & B_x \\
      -E_z & B_y & -B_x & 0
    \end{pmatrix}
  \]
  Applying the result from part one we get the following inhomogeneous Maxwell equations.
  \begin{align*}
    \nabla\cdot\vector{E} = ρ
    \separate
    -\partial_t E + \nabla\times \vector B = \vector{j}
  \end{align*}
  Because there was only one equivalent transformation the inhomogeneous Maxwell equations are indeed equivalent to the Euler-Lagrange equations.
  \[
    \left.
    \begin{aligned}
      \nabla\cdot\vector{E} &= ρ \\
      \nabla\times\vector{B} &= \vector{j} + \partial_t\vector{E}
    \end{aligned}
    \quad \right\}
    \quad
    \iff
    \quad
    \partial_n F^{mn} = j^m
  \]

  \paragraph{(3):}
  First we will look at the definition of the left hand side of the Bianchi identity.
  \[
    \partial_m\tilde{F}^{mn} = \frac{1}{2}ε^{mnpq}\roundBrackets{\partial_m\partial_pA_q - \partial_m\partial_qA_p}
  \]
  Interchanging the index $p$ with the index $m$ or the index $q$ with the index $m$ does not change the value of the left hand side.
  If we again interchange those indices in the Levi-Civita tensor then this tensor will switch its sign.
  \begin{align*}
    \partial_m\tilde{F}^{mn} &= \frac{1}{2} ε^{pnmq}\roundBrackets{\partial_p\partial_mA_q - \partial_p\partial_qA_m} = -\frac{1}{2} ε^{mnpq}\roundBrackets{\partial_p\partial_mA_q - \partial_p\partial_qA_m} \\
    \partial_m\tilde{F}^{mn} &= \frac{1}{2} ε^{qnpm}\roundBrackets{\partial_q\partial_p A_m - \partial_q\partial_m A_p} = -\frac{1}{2}ε^{mnpq}\roundBrackets{\partial_q\partial_pA_m - \partial_q\partial_mA_p}
  \end{align*}
  Because $A$ is two times continuous differentiable, we can use the theorem by Schwarz and interchange the order of the partial derivatives.
  Together with the three equations we already have, the result becomes the following.
  \begin{align*}
    \partial_m\tilde{F}^{mn} &= \frac{1}{3}\roundBrackets{\partial_m\tilde{F}^{mn} + \partial_m\tilde{F}^{mn} + \partial_m\tilde{F}^{mn}} \\
    &= \frac{1}{6}ε^{mnpq}\
      \begin{aligned}[t]
        &(\partial_m\partial_pA_q - \partial_m\partial_qA_p - \partial_p\partial_mA_q \\
        &+\partial_p\partial_qA_m - \partial_q\partial_pA_m + \partial_q\partial_mA_p)
      \end{aligned}
      \\
    &= 0
  \end{align*}

  \paragraph{(4):}
  We also know that $\tilde{F}$ is given by the following matrix.
  \[
    \roundBrackets{\tilde{F}^{mn}}_{m,n\in I_0} =
    \begin{pmatrix}
      0 & B_x & B_y & B_z \\
      -B_x & 0 & -E_z & E_y \\
      -B_y & E_z & 0 & -E_x \\
      -B_z & -E_y & E_x & 0
    \end{pmatrix}
  \]
  Applying the Bianchi identity we get the homogeneous Maxwell equations.
  \[
    -\nabla\cdot\vector{B} = 0
    \separate
    \nabla\times\vector{E} + \partial_t\vector{B} = 0
  \]
  Again we have only transformed the equations to another notation.
  The homogeneous Maxwell equations are therefore equivalent to the Bianchi identity.
  \[
    \left.
    \begin{aligned}
      \nabla\cdot\vector{B} &= 0\\
      \nabla\times\vector{E} &= - \partial_t\vector{B}
    \end{aligned}
    \quad
    \right\}
    \quad
    \iff
    \quad
    \partial_m \tilde{F}^{mn} = 0
  \]

  \paragraph{(5):}
  To deduce the the covariant continuity equation we will again rely on the theorem by Schwarz to interchange the order of the partial derivatives.
  \begin{align*}
    \partial_n j^n &= \partial_n\partial_m F^{nm} \\
    &= \partial_n\partial_m\partial^n A^m - \partial_n\partial_m\partial^m A^n \\
    &= \partial_n\partial^n\partial_m A^m - \partial_m\partial^m\partial_n A^n \\
    &= \partial_n\partial^n\partial_m A^m - \partial_n\partial^n\partial_m A^m \\
    &= 0
  \end{align*}

  \paragraph{(6):}
  Let $\function{χ}{\setReal^4}{\setReal}$ be a two times continuous differentiable function.
  Then we define the transformed potentials $A'_n$.
  \[
    A'_n = A_n + \partial_n χ
  \]
  Next we will compute the field strength tensor of the transformed potentials again based on Schwarz's theorem.
  We will see it is invariant under this transformation.
  \begin{align*}
    F'_{mn} &= \partial_m A'_n - \partial_n A'_m \\
    &= \partial_m \roundBrackets{A_n + \partial_n χ} - \partial_n \roundBrackets{A_m + \partial_m χ} \\
    &= \partial_m A_n - \partial_n A_m + \partial_m\partial_n χ - \partial_n\partial_m χ \\
    &= F_{mn} + \partial_m\partial_n χ - \partial_m\partial_n χ \\
    &= F_{mn}
  \end{align*}
  Hence, the kinetic part of the Lagrangian is invariant under this transformation, too.
  For the potential part of the Lagrangian we get the following.
  \begin{align*}
    j^mA'_m &= j^m \roundBrackets{A_m + \partial_m χ} \\
    &= j^m A_m + j^m\partial_m χ
  \end{align*}
  Here the function $j^m\partial_m χ$ does not depend on any potential.
  But we know that we can add an arbitrary function to the Lagrangian without changing the Euler-Lagrange equations.
  Therefore the Euler-Lagrange equations for the transformed potentials are the same.
  For these equations we already showed the continuity equation.
\end{document}