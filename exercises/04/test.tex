\documentclass[a4paper,fleqn]{article}
\usepackage{standard}
\geometry{top=30mm,bottom=30mm,left=42mm,right=42mm}

\begin{document}
  \hrule
  \medskip
  \begin{center}
    \Large
    \textbf{Quantum Field Theory \\ Exercise Sheet 4}
  \end{center}
  \medskip
  \begin{minipage}[t]{0.45\textwidth}
    \begin{raggedleft}
      Markus Pawellek\\
      markuspawellek@gmail.com\\
    \end{raggedleft}
  \end{minipage}
  \hfill
  \begin{minipage}[t]{0.45\textwidth}
    Clemens Anschütz\\
    clemens.anschuetz@uni-jena.de
  \end{minipage}
  \medskip
  \hrule
  \bigskip

  \section*{Problem 9: Lorentz transformations} % (fold)
  \label{sec:problem_9_lorentz_transformations}

    If not defined otherwise we will use the Latin alphabet $\set{a,b,c,\ldots}{}$ as a set of dummy indices in the set $I_0\define\set{0,1,2,3}{}$ and the Greek alphabet $\set{α,β,γ,\ldots}{}$ as a set of dummy indices in the set $I\define\set{1,2,3}{}$ for the sum convention of Einstein.
    We denote the Minkowski metric as follows.
    \[
      η \define \roundBrackets{η_{mn}} \define \roundBrackets{η^{mn}} \define \diagonalMatrix{-1,1,1,1}
    \]
    \newcommand{\lorentzGroup}{\mathds{L}}
    We define the Lorentz group $\lorentzGroup$ as a subset of invertible matrices.
    \[
      \lorentzGroup \define \set{L \in \mathrm{GL}_4(\setReal)}{\transpose{L}ηL = η}
    \]

    \paragraph{(1):}
    Let $L\in\lorentzGroup$ with the following definition for $ω\in\setReal^{4\times 4}$.
    We say $\infinitesimal{L}$ is the infinitesimal change of $L$.
    \[
      L\indices{^m_n} = δ\indices{^m_n}
      \separate
      \infinitesimal{L}\indices{^m_n} = ω\indices{^m_n}
    \]
    From the definition of the Lorentz group we do know that the following condition has to hold.
    \[
      η\indices{_{mn}} = η\indices{_{pq}} L\indices{^p_m} L\indices{^q_n}
    \]
    By computing the infinitesimal change, we can directly proof the lemma.
    \begin{align*}
      0 &= η\indices{_{pq}} \roundBrackets{ \infinitesimal{L}\indices{^p_m} L\indices{^q_n} + L\indices{^p_m}\infinitesimal{L}\indices{^q_n} } \\
      &= η\indices{_{pq}} δ\indices{^q_n} ω\indices{^p_m} + η\indices{_{pq}} δ\indices{^p_m} ω\indices{^q_n} \\
      &= η\indices{_{pn}} ω\indices{^p_m} + η\indices{_{mq}} ω\indices{^q_n} \\
      &= ω\indices{_{nm}} + ω\indices{_{mn}}
    \end{align*}

    \paragraph{(2):}
    Let $L\in\lorentzGroup$ then we do know from the definition of $\lorentzGroup$
    \[
      η = \transpose{L}η L
    \]
    Now we use some rules for the computation of determinants.
    The determinant of matrix product is the same as the product of determinants of these matrices.
    Furthermore the determinant does not change under transposing the given matrix.
    \begin{align*}
      \det η &= \det \roundBrackets{\transpose{L}η L} \\
      &= \det\transpose{L} \cdot \det η \cdot \det L \\
      &= \det η\cdot\roundBrackets{\det L}^2
    \end{align*}
    We divide this equation by $\det η$ and get the following result.
    \[
      \roundBrackets{\det L}^2 = 1
      \quad \implies \quad
      \det L = \pm 1
    \]

    Again we use the definition of $\lorentzGroup$.
    \[
      η\indices{_{mn}} = η\indices{_{pq}} L\indices{^p_m} L\indices{^q_n}
    \]
    We can specialize this.
    \begin{align*}
      η_{00} &= η_{pq} L\indices{^p_0} L\indices{^q_0} \\
      &= η_{00}\roundBrackets{L\indices{^0_0}}^2 + 2 η_{α0} L\indices{^α_0} L\indices{^0_0} + η_{αβ} L\indices{^α_0} L\indices{^β_0}
    \end{align*}

    % \paragraph{(3):}

  % section problem_9_lorentz_transformations (end)

\end{document}